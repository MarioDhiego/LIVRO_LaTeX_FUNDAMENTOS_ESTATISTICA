\chapter*{Prefácio}

\inic A idéia de escrever um texto introdutório sobre fundamentos de estatística para gestão pública surgiu da necessidade de se divulgar o potencial dessa teoria tanto no
seu aspecto estatístico-matemático quanto na sua aplicação e
interpretação em diversas áreas do setor público.\vst

A importância da estatística para o gestor público pode ser vista através da sua utilização ao nível do Estado, de Organizações Sociais e Profissionais, do cidadão comum e ao nível acadêmico. Não restam dúvidas de que uma base de informações qualificada é fundamental para a adequada gestão das políticas públicas.\vst  

A estatística fornece ferramentas importantes para que os governos possam definir melhor suas metas, avaliar sua performance, identificar seus pontos fortes e fracos e atuar na melhoria contínua das políticas públicas.
\vst

Nossa maior preocupação foi a de escrever um texto que pudesse ser
utilizado não só pelos estatísticos, mas também por profissionais de outras áreas. O sucesso da Estatística passa necessariamente
pelo trabalho conjunto de especialistas dessas várias áreas. \vst

Muito do material e idéias apresentadas nesse livro foram
desenvolvidos durante o planejamento e a análie de diversas experiências Técnicas e Acadêmicas, ao longo de 18 anos trabalhando na Gestão Pública Estadual.
\vst 


Devido a enorme abrangência da Estatística, procura-se detalhar os pontos que achamos mais interessantes para um texto introdutório e
fornecer o maior número possível de referências bibliográficas que
cobrissem os outros pontos.\vst

O profissional que domina os princípios básicos estatísticos tem em suas mãos uma poderosa ferramenta que poderá ser uma aliada ao longo da carreira, as aplicações são diversas. Assim, compreender esse contexto é a primeira lição de \textbf{Estatística}.
\vst

\newpage
O livro tem finalidade didática, sem a preocupação com o aprofundamento dos assuntos, o que provavelmente afastaria os estudantes iniciantes no assunto.\vst 


O livro foi escrito utilizando linguagem de programação open source chamada \LaTeX (é um conjunto de comandos adicionais (macros) para o \TeX, elaborado em meados da década de 80 por \textbf{Leslie Lamport}. Em sua modernidade, utilizou-se um ambiente de desenvolvimento integrado para manipulação dos scripts, chamado \textbf{OvearLeaf}.
\vst

Para facilitar a análise dos dados e a construção dos gráficos foram introduzidos vários exemplos elaborados com os recursos computacionais do software Estatístico \textbf{R 4.2.2} versão para Windows. 
\vst

%Este trabalho foi parcialmente financiado pelo DETRAN-PA.

\vst

\begin{centering}

\vst

Setembro 2023 
\vsm

Mário Diego Rocha Valente (Analista de Trânsito/Detran-PA) \\

Geovana Raio Pires (Técnica em Gestão Pública/Seplad-PA)\\

Héliton Ribeiro Tavares (Prof Dr. Associado/UFPA)\\

Waldenei Travassos De Queiroz (Prof Dr. Titular/UFRA)\\



\end{centering}
